
\section{Fuzzy Prolog}

Prolog is based on First Order Logic, where the truth values are only
{\em true} and {\em false}. In Fuzzy Logic, there is a whole range of
truth values: any real from 0 (absolute falsity) to 1 (absolute
truth). Fuzzy Prolog is a logic programming language based on
this logic.

A (logic) concept can be seen as a set which includes the elements of 
``universe'' to which the concept is applicable. A predicate then can
be associates with a {\em characteristic function} which determines if
the arguments given to the predicates are in the set defining the
predicate (satisfy the relation represented by the predicate) or not.
%
Given a relevant universal set $U$, a (classic) set $A$, associated to
a predicate, can be defined by a characteristic function 
$A : U\rightarrow \{0,1\}$. If an element $x$ in $U$ belongs to $A$
then $A(x) = 1$, otherwise $A(x) = 0$.

Fuzzy sets, on the other hand, try to model imprecise concepts. An
element in the universal set belongs to a fuzzy sets in a certain
grade. For example, the youth of a person given his age. A person
which is 25 years old is certainly young, and if he is 50 then he is
not young, but if he is 36 years old he is young in some
grade. Therefore, a fuzzy set $A$ is defined by a characteristic
function $A : X \rightarrow [0,1]$. Figure \ref{fig:fuzzysetyoung}
shows the characteristic function for the fuzzy set for \texttt{young}.
The value $A(x)$ is the membership value of $x$ in the fuzzy set $A$,
if we consider $A$ as a predicate, this value is called the truth value.

\begin{figure}[htbp]
  \begin{center}
\setlength{\unitlength}{2547sp}%
%
\begingroup\makeatletter\ifx\SetFigFont\undefined%
\gdef\SetFigFont#1#2#3#4#5{%
  \reset@font\fontsize{#1}{#2pt}%
  \fontfamily{#3}\fontseries{#4}\fontshape{#5}%
  \selectfont}%
\fi\endgroup%
\begin{picture}(5175,2685)(1051,-3136)
\thinlines
{\color[rgb]{0,0,0}\put(1801,-2761){\vector( 0, 1){2250}}
\put(1801,-2761){\vector( 1, 0){4125}}
}%
{\color[rgb]{0,0,0}\put(2401,-2761){\line( 0,-1){150}}
}%
{\color[rgb]{0,0,0}\put(3601,-2761){\line( 0,-1){150}}
}%
{\color[rgb]{0,0,0}\put(3001,-2761){\line( 0,-1){150}}
}%
{\color[rgb]{0,0,0}\put(4801,-2761){\line( 0,-1){150}}
}%
{\color[rgb]{0,0,0}\put(5401,-2761){\line( 0,-1){150}}
}%
{\color[rgb]{0,0,0}\put(1801,-961){\line(-1, 0){150}}
}%
\thicklines
{\color[rgb]{0,0,0}\put(1801,-2761){\line( 1, 0){600}}
\put(2401,-2761){\line( 1, 2){900}}
\put(3301,-961){\line( 1, 0){600}}
\put(3901,-961){\line( 1,-2){900}}
\put(4801,-2761){\line( 1, 0){675}}
}%
\thinlines
{\color[rgb]{0,0,0}\put(4201,-2761){\line( 0,-1){150}}
}%
{\color[rgb]{0,0,0}\put(1801,-1861){\line(-1, 0){150}}
}%
\put(2326,-3136){\makebox(0,0)[lb]{\smash{\SetFigFont{8}{14.4}{\rmdefault}{\mddefault}{\updefault}{\color[rgb]{0,0,0}10}%
}}}
\put(2926,-3136){\makebox(0,0)[lb]{\smash{\SetFigFont{8}{14.4}{\rmdefault}{\mddefault}{\updefault}{\color[rgb]{0,0,0}20}%
}}}
\put(4726,-3136){\makebox(0,0)[lb]{\smash{\SetFigFont{8}{14.4}{\rmdefault}{\mddefault}{\updefault}{\color[rgb]{0,0,0}50}%
}}}
\put(5326,-3136){\makebox(0,0)[lb]{\smash{\SetFigFont{8}{14.4}{\rmdefault}{\mddefault}{\updefault}{\color[rgb]{0,0,0}60}%
}}}
\put(3526,-3136){\makebox(0,0)[lb]{\smash{\SetFigFont{8}{14.4}{\rmdefault}{\mddefault}{\updefault}{\color[rgb]{0,0,0}30}%
}}}
\put(4126,-3136){\makebox(0,0)[lb]{\smash{\SetFigFont{8}{14.4}{\rmdefault}{\mddefault}{\updefault}{\color[rgb]{0,0,0}40}%
}}}
\put(1501,-2836){\makebox(0,0)[lb]{\smash{\SetFigFont{8}{16.8}{\rmdefault}{\mddefault}{\updefault}{\color[rgb]{0,0,0}0}%
}}}
\put(1501,-1036){\makebox(0,0)[lb]{\smash{\SetFigFont{8}{16.8}{\rmdefault}{\mddefault}{\updefault}{\color[rgb]{0,0,0}1}%
}}}
\put(1276,-1936){\makebox(0,0)[lb]{\smash{\SetFigFont{8}{16.8}{\rmdefault}{\mddefault}{\updefault}{\color[rgb]{0,0,0}0.5}%
}}}
\put(6226,-2986){\makebox(0,0)[lb]{\smash{\SetFigFont{8}{14.4}{\rmdefault}{\mddefault}{\updefault}{\color[rgb]{0,0,0}Age}%
}}}
\put(1051,-586){\makebox(0,0)[lb]{\smash{\SetFigFont{8}{14.4}{\rmdefault}{\mddefault}{\updefault}{\color[rgb]{0,0,0}Youth}%
}}}
\end{picture}
    \caption{The fuzzy set \texttt{young}}
    \label{fig:fuzzysetyoung}
  \end{center}
\end{figure}

The package \texttt{fuzzy} allows Ciao to interpret fuzzy
predicates. Fuzzy predicates which correspond to fuzzy sets on a
universal set of numbers and with characteristic functions which are
polygonal can be defined by the operator 
\texttt{:\# fuzzy\_predicate}. For example, the above fuzzy set
\texttt{young} is defined as: 
\begin{quote}
\begin{verbatim}
:- use_package([...,fuzzy,...]).

young :# fuzzy_predicate [(0,0),(10,0),(25,1),(35,1),(50,0),(60,0)].
\end{verbatim}
\end{quote}
%
where each element in the list is a vertex in the polygonal function.
This sentence defines a fuzzy predicate \texttt{young/2}, where the
first argument is the age and the second argument represents the truth
value. Truth values are assigned to the variable in the last argument
as constraints (Section~\ref{sec:constr-logic-progr}) over real
numbers. For example,

\begin{verbatim}
?- young(30,M).

M = 1 ? 

yes
?- young(38,M).

(M.=.0.8 ? 

yes
?- young(45,M).

M.=.0.333333333333333313 ? 

yes
?- young(50,M).

M = 0 ? 
\end{verbatim}

More complex fuzzy predicates can be defined by fuzzy clauses. We use
\texttt{:{\tiny$^\sim$}} instead of \texttt{:-} as the neck of
fuzzy clauses to distinguish them from Prolog clauses. For example,
\begin{quote}
{\tt
\begin{tabbing} 
young\_co\=uple(X,Y,Mu) :{\tiny$^\sim$} \\
\>        age(X,X1), \\
\>        age(Y,Y1),  \\
\>        young(X1,MuX), \\
\>        young(Y1,MuY). \\
\\
age(john,46). \\
age(rose,36).
\end{tabbing}
}
\end{quote}
%
defines a fuzzy predicate \texttt{young\_couple/3} that uses in its
definition a fuzzy predicate \texttt{young/2} and a Prolog predicate
\texttt{age/2}. The truth value of Prolog calls in the body of the
fuzzy clause is 1 is it succeeds and 0 otherwise. The truth value of
\texttt{young\_couple} is obtained by aggregating the truth values of
the calls in the body. By default the aggregation function is
\texttt{min}, i.e., the minimum from all body goals. In this example,

\begin{verbatim}
?- young_couple(john,rose,M).

M.=.0.26 ? 
\end{verbatim}
%
because $0.26$ is the minimum of $\{1,1,0.93,0.26\}$ which are the
truth values of the calls \texttt{age(john,X1)},
\texttt{age(rose,Y1)}, \texttt{young(46,MuX)} and
\texttt{young(36,MuY)} respectively.

Aggregation functions are indicated after the operator
\texttt{:{\tiny$^\sim$}}. For example, if we want the truth value of
\texttt{young\_couple/3} to be the product of the truth values in its
definition then we write: 
\begin{quote}
{\tt
\begin{tabbing} 
young\_co\=uple(X,Y,Mu) :{\tiny$^\sim$} prod \\
\>        age(X,X1), \\
\>        age(Y,Y1),  \\
\>        young(X1,MuX), \\
\>        young(Y1,MuY).
\end{tabbing}
}
\end{quote}
%
so that now:

\begin{verbatim}
?- young_couple(john,rose,M).

M.=.0.23 ? 
\end{verbatim}
%
since $0.23$ is the product of $\{1,1,0.93,0.26\}$. 

Other {\em aggregations functions} are:
\begin{itemize}
\item \texttt{max}: the maximum truth value in the body of the
  clause. It is a ``disjunctive'' aggregation operation.
\item \texttt{luka}: the Lukasiewicz T-norm. It is a ``conjunctive''
  operator like \texttt{min} and \texttt{prod}. The Lukasiewicz T-norm
  of two values $x$ and $y$ is defined as $\max(0,x + y -1)$.
\item \texttt{dluka}: the Lukasiewicz T-conorm. It is defined as
  $\min(1,x + y)$. 
\item \texttt{dprod} the product T-conorm. It is defined as $x + y -
  (x * y)$.
\end{itemize}
%
(T-conorms are disjunctive operators dual to the conjunctive operator
of the corresponding T-norm.)

The min-max fuzzy logic can be modeled using the operators \texttt{max} and
\texttt{min}. In the same way, using the operators \texttt{luka} and
\texttt{dluka} the lukasiewicz fuzzy logic can be modeled. Let us see
an example. 

Suppose we want to measure which is the possibility that a couple
of values, obtained throwing two loaded dice, sum 5. Let us suppose we
only know that one die is loaded to obtain a small value and the other
is loaded to obtain a large value. The program to declare this is the
following: 
\begin{quote}
{\tt
\begin{tabbing} 
small :\# fuzzy\_predicate([(1,1),(2,1),(3,0.7),(4,0.3),(5,0),(6,0)]). \\
large :\# fuzzy\_predicate([(1,0),(2,0),(3,0.3),(4,0.7),(5,1),(6,1)]). \\
\ \\
die1(X,M)\= 123456789012345678901234567890\= two\_dice\=\kill
die1(X,M)\> :{\tiny$^\sim$}       \> two\_dice\=(X,Y,M) :{\tiny$^\sim$} min \\
\>       small(X,M).            \> \> die1(X,M1), \\
\>                              \> \> die2(Y,M2). \\
die2(X,M) :{\tiny$^\sim$} \\
\>        large(X,M).
\end{tabbing}
}
\end{quote}
%
where \texttt{two\_dice(X,Y,M)} gives the possibility that the first die
obtains the value \texttt{X} and the second the value \texttt{Y}.
Therefore in this clause we need a conjunctive operator as
\texttt{min}. Now we can define:
\begin{quote}
{\tt
\begin{tabbing} 
sum(5,M)\= :{\tiny$^\sim$} max \\
\>        two\_dice(4,1,M1), \\
\>        two\_dice(1,4,M2), \\
\>        two\_dice(3,2,M3), \\
\>        two\_dice(2,3,M4).
\end{tabbing}
}
\end{quote}
%
for which we need a disjunctive operator as \texttt{max} which
aggregates the different ways of obtaining 5 for the sum. We will
get for example: 

\begin{verbatim}
?- sum(5,M).

M.=.0.7 ?
\end{verbatim}

The package also gives the possibility of defining fuzzy predicates
which are the fuzzy negation of other fuzzy predicates. For example,
the fuzzy predicate \texttt{small} could be defined as:
\begin{quote}
\begin{verbatim}
small :# fnot large/2
\end{verbatim}
\end{quote}
%
so that the truth value \verb+M+ of \verb+small(X,M)+ is defined as
\verb+1-N+ where \verb+N+ is the truth value of \verb+large(X,N)+.  

It is possible to ``fuzzify'' Prolog predicates. For
example to fuzzify \verb+age/2+ it is only necessary to write:
\begin{quote}
\begin{verbatim}
age_f :# fuzzy age/2
\end{verbatim}
\end{quote}
%
and the program is expanded with a new fuzzy predicate
\texttt{age\_f/3} (the last argument is the truth value) with truth
value equal to 0 if \texttt{age/2} fails and 1 otherwise. 

Prolog calls
in bodies of fuzzy clauses are automatically fuzzified, except those
that are enclosed in \verb+{}+ brackets. This is useful if we want
truly prolog calls to define a fuzzy predicate, for example to set a
threshold for a fuzzy clause like in the following program:
\begin{quote}
{\tt
\begin{tabbing} 
very\_goo\=d\_credit\_customer(X,Mu):{\tiny$^\sim$} min \\
\>        good\_credit\_customer(X,M), \\
\>        \{Mu .>=. 0.8\}.  \\
very\_good\_credit\_customer(X,Mu):{\tiny$^\sim$} min \\
\>        good\_credit\_customer(X,M), \\
\>        fail, \\
\>        \{Mu .<. 0.8\}.
\end{tabbing}
}
\end{quote}
%
which defines a very good credit customer as a good credit customer
with truth value greater than $0.8$. A  good credit customer is
defined, for example, as follows:
\begin{quote}
{\tt
\begin{tabbing} 
good\_credit\_customer(john,0.8):{\tiny$^\sim$} . \\
good\_credit\_customer(mary,0.5):{\tiny$^\sim$} . \\
good\_credit\_customer(alan,0.9):{\tiny$^\sim$} . 
\end{tabbing}
}
\end{quote}

Note how fuzzy facts do require the operator \texttt{:{\tiny$^\sim$}}
(with a space separating it from the final dot). Otherwise, they will
be deemed as normal Prolog clauses. Mixing fuzzy clauses and Prolog
clauses may cause unexpected behaviours. The compiler will signal
the problem with a warning message indicating discontiguous clauses for
the predicate.

In this case we obtain the following answers: 

\begin{verbatim}
?- very_good_credit_customer(X,Mu).

X = john,
Mu.=.0.8 ? ;

X = alan,
Mu.=.0.9 ? ;

Mu = 0,
X = mary ? ;

no
?- 
\end{verbatim}
